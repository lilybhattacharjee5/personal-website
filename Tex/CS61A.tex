
% Default to the notebook output style

    


% Inherit from the specified cell style.




    
\documentclass[11pt]{article}

    
    
    \usepackage[T1]{fontenc}
    % Nicer default font (+ math font) than Computer Modern for most use cases
    \usepackage{mathpazo}

    % Basic figure setup, for now with no caption control since it's done
    % automatically by Pandoc (which extracts ![](path) syntax from Markdown).
    \usepackage{graphicx}
    % We will generate all images so they have a width \maxwidth. This means
    % that they will get their normal width if they fit onto the page, but
    % are scaled down if they would overflow the margins.
    \makeatletter
    \def\maxwidth{\ifdim\Gin@nat@width>\linewidth\linewidth
    \else\Gin@nat@width\fi}
    \makeatother
    \let\Oldincludegraphics\includegraphics
    % Set max figure width to be 80% of text width, for now hardcoded.
    \renewcommand{\includegraphics}[1]{\Oldincludegraphics[width=.8\maxwidth]{#1}}
    % Ensure that by default, figures have no caption (until we provide a
    % proper Figure object with a Caption API and a way to capture that
    % in the conversion process - todo).
    \usepackage{caption}
    \DeclareCaptionLabelFormat{nolabel}{}
    \captionsetup{labelformat=nolabel}

    \usepackage{adjustbox} % Used to constrain images to a maximum size 
    \usepackage{xcolor} % Allow colors to be defined
    \usepackage{enumerate} % Needed for markdown enumerations to work
    \usepackage{geometry} % Used to adjust the document margins
    \usepackage{amsmath} % Equations
    \usepackage{amssymb} % Equations
    \usepackage{textcomp} % defines textquotesingle
    % Hack from http://tex.stackexchange.com/a/47451/13684:
    \AtBeginDocument{%
        \def\PYZsq{\textquotesingle}% Upright quotes in Pygmentized code
    }
    \usepackage{upquote} % Upright quotes for verbatim code
    \usepackage{eurosym} % defines \euro
    \usepackage[mathletters]{ucs} % Extended unicode (utf-8) support
    \usepackage[utf8x]{inputenc} % Allow utf-8 characters in the tex document
    \usepackage{fancyvrb} % verbatim replacement that allows latex
    \usepackage{grffile} % extends the file name processing of package graphics 
                         % to support a larger range 
    % The hyperref package gives us a pdf with properly built
    % internal navigation ('pdf bookmarks' for the table of contents,
    % internal cross-reference links, web links for URLs, etc.)
    \usepackage{hyperref}
    \usepackage{longtable} % longtable support required by pandoc >1.10
    \usepackage{booktabs}  % table support for pandoc > 1.12.2
    \usepackage[inline]{enumitem} % IRkernel/repr support (it uses the enumerate* environment)
    \usepackage[normalem]{ulem} % ulem is needed to support strikethroughs (\sout)
                                % normalem makes italics be italics, not underlines
    

    
    
    % Colors for the hyperref package
    \definecolor{urlcolor}{rgb}{0,.145,.698}
    \definecolor{linkcolor}{rgb}{.71,0.21,0.01}
    \definecolor{citecolor}{rgb}{.12,.54,.11}

    % ANSI colors
    \definecolor{ansi-black}{HTML}{3E424D}
    \definecolor{ansi-black-intense}{HTML}{282C36}
    \definecolor{ansi-red}{HTML}{E75C58}
    \definecolor{ansi-red-intense}{HTML}{B22B31}
    \definecolor{ansi-green}{HTML}{00A250}
    \definecolor{ansi-green-intense}{HTML}{007427}
    \definecolor{ansi-yellow}{HTML}{DDB62B}
    \definecolor{ansi-yellow-intense}{HTML}{B27D12}
    \definecolor{ansi-blue}{HTML}{208FFB}
    \definecolor{ansi-blue-intense}{HTML}{0065CA}
    \definecolor{ansi-magenta}{HTML}{D160C4}
    \definecolor{ansi-magenta-intense}{HTML}{A03196}
    \definecolor{ansi-cyan}{HTML}{60C6C8}
    \definecolor{ansi-cyan-intense}{HTML}{258F8F}
    \definecolor{ansi-white}{HTML}{C5C1B4}
    \definecolor{ansi-white-intense}{HTML}{A1A6B2}

    % commands and environments needed by pandoc snippets
    % extracted from the output of `pandoc -s`
    \providecommand{\tightlist}{%
      \setlength{\itemsep}{0pt}\setlength{\parskip}{0pt}}
    \DefineVerbatimEnvironment{Highlighting}{Verbatim}{commandchars=\\\{\}}
    % Add ',fontsize=\small' for more characters per line
    \newenvironment{Shaded}{}{}
    \newcommand{\KeywordTok}[1]{\textcolor[rgb]{0.00,0.44,0.13}{\textbf{{#1}}}}
    \newcommand{\DataTypeTok}[1]{\textcolor[rgb]{0.56,0.13,0.00}{{#1}}}
    \newcommand{\DecValTok}[1]{\textcolor[rgb]{0.25,0.63,0.44}{{#1}}}
    \newcommand{\BaseNTok}[1]{\textcolor[rgb]{0.25,0.63,0.44}{{#1}}}
    \newcommand{\FloatTok}[1]{\textcolor[rgb]{0.25,0.63,0.44}{{#1}}}
    \newcommand{\CharTok}[1]{\textcolor[rgb]{0.25,0.44,0.63}{{#1}}}
    \newcommand{\StringTok}[1]{\textcolor[rgb]{0.25,0.44,0.63}{{#1}}}
    \newcommand{\CommentTok}[1]{\textcolor[rgb]{0.38,0.63,0.69}{\textit{{#1}}}}
    \newcommand{\OtherTok}[1]{\textcolor[rgb]{0.00,0.44,0.13}{{#1}}}
    \newcommand{\AlertTok}[1]{\textcolor[rgb]{1.00,0.00,0.00}{\textbf{{#1}}}}
    \newcommand{\FunctionTok}[1]{\textcolor[rgb]{0.02,0.16,0.49}{{#1}}}
    \newcommand{\RegionMarkerTok}[1]{{#1}}
    \newcommand{\ErrorTok}[1]{\textcolor[rgb]{1.00,0.00,0.00}{\textbf{{#1}}}}
    \newcommand{\NormalTok}[1]{{#1}}
    
    % Additional commands for more recent versions of Pandoc
    \newcommand{\ConstantTok}[1]{\textcolor[rgb]{0.53,0.00,0.00}{{#1}}}
    \newcommand{\SpecialCharTok}[1]{\textcolor[rgb]{0.25,0.44,0.63}{{#1}}}
    \newcommand{\VerbatimStringTok}[1]{\textcolor[rgb]{0.25,0.44,0.63}{{#1}}}
    \newcommand{\SpecialStringTok}[1]{\textcolor[rgb]{0.73,0.40,0.53}{{#1}}}
    \newcommand{\ImportTok}[1]{{#1}}
    \newcommand{\DocumentationTok}[1]{\textcolor[rgb]{0.73,0.13,0.13}{\textit{{#1}}}}
    \newcommand{\AnnotationTok}[1]{\textcolor[rgb]{0.38,0.63,0.69}{\textbf{\textit{{#1}}}}}
    \newcommand{\CommentVarTok}[1]{\textcolor[rgb]{0.38,0.63,0.69}{\textbf{\textit{{#1}}}}}
    \newcommand{\VariableTok}[1]{\textcolor[rgb]{0.10,0.09,0.49}{{#1}}}
    \newcommand{\ControlFlowTok}[1]{\textcolor[rgb]{0.00,0.44,0.13}{\textbf{{#1}}}}
    \newcommand{\OperatorTok}[1]{\textcolor[rgb]{0.40,0.40,0.40}{{#1}}}
    \newcommand{\BuiltInTok}[1]{{#1}}
    \newcommand{\ExtensionTok}[1]{{#1}}
    \newcommand{\PreprocessorTok}[1]{\textcolor[rgb]{0.74,0.48,0.00}{{#1}}}
    \newcommand{\AttributeTok}[1]{\textcolor[rgb]{0.49,0.56,0.16}{{#1}}}
    \newcommand{\InformationTok}[1]{\textcolor[rgb]{0.38,0.63,0.69}{\textbf{\textit{{#1}}}}}
    \newcommand{\WarningTok}[1]{\textcolor[rgb]{0.38,0.63,0.69}{\textbf{\textit{{#1}}}}}
    
    
    % Define a nice break command that doesn't care if a line doesn't already
    % exist.
    \def\br{\hspace*{\fill} \\* }
    % Math Jax compatability definitions
    \def\gt{>}
    \def\lt{<}
    % Document parameters
    \title{CS61A}
    
    
    

    % Pygments definitions
    
\makeatletter
\def\PY@reset{\let\PY@it=\relax \let\PY@bf=\relax%
    \let\PY@ul=\relax \let\PY@tc=\relax%
    \let\PY@bc=\relax \let\PY@ff=\relax}
\def\PY@tok#1{\csname PY@tok@#1\endcsname}
\def\PY@toks#1+{\ifx\relax#1\empty\else%
    \PY@tok{#1}\expandafter\PY@toks\fi}
\def\PY@do#1{\PY@bc{\PY@tc{\PY@ul{%
    \PY@it{\PY@bf{\PY@ff{#1}}}}}}}
\def\PY#1#2{\PY@reset\PY@toks#1+\relax+\PY@do{#2}}

\expandafter\def\csname PY@tok@w\endcsname{\def\PY@tc##1{\textcolor[rgb]{0.73,0.73,0.73}{##1}}}
\expandafter\def\csname PY@tok@c\endcsname{\let\PY@it=\textit\def\PY@tc##1{\textcolor[rgb]{0.25,0.50,0.50}{##1}}}
\expandafter\def\csname PY@tok@cp\endcsname{\def\PY@tc##1{\textcolor[rgb]{0.74,0.48,0.00}{##1}}}
\expandafter\def\csname PY@tok@k\endcsname{\let\PY@bf=\textbf\def\PY@tc##1{\textcolor[rgb]{0.00,0.50,0.00}{##1}}}
\expandafter\def\csname PY@tok@kp\endcsname{\def\PY@tc##1{\textcolor[rgb]{0.00,0.50,0.00}{##1}}}
\expandafter\def\csname PY@tok@kt\endcsname{\def\PY@tc##1{\textcolor[rgb]{0.69,0.00,0.25}{##1}}}
\expandafter\def\csname PY@tok@o\endcsname{\def\PY@tc##1{\textcolor[rgb]{0.40,0.40,0.40}{##1}}}
\expandafter\def\csname PY@tok@ow\endcsname{\let\PY@bf=\textbf\def\PY@tc##1{\textcolor[rgb]{0.67,0.13,1.00}{##1}}}
\expandafter\def\csname PY@tok@nb\endcsname{\def\PY@tc##1{\textcolor[rgb]{0.00,0.50,0.00}{##1}}}
\expandafter\def\csname PY@tok@nf\endcsname{\def\PY@tc##1{\textcolor[rgb]{0.00,0.00,1.00}{##1}}}
\expandafter\def\csname PY@tok@nc\endcsname{\let\PY@bf=\textbf\def\PY@tc##1{\textcolor[rgb]{0.00,0.00,1.00}{##1}}}
\expandafter\def\csname PY@tok@nn\endcsname{\let\PY@bf=\textbf\def\PY@tc##1{\textcolor[rgb]{0.00,0.00,1.00}{##1}}}
\expandafter\def\csname PY@tok@ne\endcsname{\let\PY@bf=\textbf\def\PY@tc##1{\textcolor[rgb]{0.82,0.25,0.23}{##1}}}
\expandafter\def\csname PY@tok@nv\endcsname{\def\PY@tc##1{\textcolor[rgb]{0.10,0.09,0.49}{##1}}}
\expandafter\def\csname PY@tok@no\endcsname{\def\PY@tc##1{\textcolor[rgb]{0.53,0.00,0.00}{##1}}}
\expandafter\def\csname PY@tok@nl\endcsname{\def\PY@tc##1{\textcolor[rgb]{0.63,0.63,0.00}{##1}}}
\expandafter\def\csname PY@tok@ni\endcsname{\let\PY@bf=\textbf\def\PY@tc##1{\textcolor[rgb]{0.60,0.60,0.60}{##1}}}
\expandafter\def\csname PY@tok@na\endcsname{\def\PY@tc##1{\textcolor[rgb]{0.49,0.56,0.16}{##1}}}
\expandafter\def\csname PY@tok@nt\endcsname{\let\PY@bf=\textbf\def\PY@tc##1{\textcolor[rgb]{0.00,0.50,0.00}{##1}}}
\expandafter\def\csname PY@tok@nd\endcsname{\def\PY@tc##1{\textcolor[rgb]{0.67,0.13,1.00}{##1}}}
\expandafter\def\csname PY@tok@s\endcsname{\def\PY@tc##1{\textcolor[rgb]{0.73,0.13,0.13}{##1}}}
\expandafter\def\csname PY@tok@sd\endcsname{\let\PY@it=\textit\def\PY@tc##1{\textcolor[rgb]{0.73,0.13,0.13}{##1}}}
\expandafter\def\csname PY@tok@si\endcsname{\let\PY@bf=\textbf\def\PY@tc##1{\textcolor[rgb]{0.73,0.40,0.53}{##1}}}
\expandafter\def\csname PY@tok@se\endcsname{\let\PY@bf=\textbf\def\PY@tc##1{\textcolor[rgb]{0.73,0.40,0.13}{##1}}}
\expandafter\def\csname PY@tok@sr\endcsname{\def\PY@tc##1{\textcolor[rgb]{0.73,0.40,0.53}{##1}}}
\expandafter\def\csname PY@tok@ss\endcsname{\def\PY@tc##1{\textcolor[rgb]{0.10,0.09,0.49}{##1}}}
\expandafter\def\csname PY@tok@sx\endcsname{\def\PY@tc##1{\textcolor[rgb]{0.00,0.50,0.00}{##1}}}
\expandafter\def\csname PY@tok@m\endcsname{\def\PY@tc##1{\textcolor[rgb]{0.40,0.40,0.40}{##1}}}
\expandafter\def\csname PY@tok@gh\endcsname{\let\PY@bf=\textbf\def\PY@tc##1{\textcolor[rgb]{0.00,0.00,0.50}{##1}}}
\expandafter\def\csname PY@tok@gu\endcsname{\let\PY@bf=\textbf\def\PY@tc##1{\textcolor[rgb]{0.50,0.00,0.50}{##1}}}
\expandafter\def\csname PY@tok@gd\endcsname{\def\PY@tc##1{\textcolor[rgb]{0.63,0.00,0.00}{##1}}}
\expandafter\def\csname PY@tok@gi\endcsname{\def\PY@tc##1{\textcolor[rgb]{0.00,0.63,0.00}{##1}}}
\expandafter\def\csname PY@tok@gr\endcsname{\def\PY@tc##1{\textcolor[rgb]{1.00,0.00,0.00}{##1}}}
\expandafter\def\csname PY@tok@ge\endcsname{\let\PY@it=\textit}
\expandafter\def\csname PY@tok@gs\endcsname{\let\PY@bf=\textbf}
\expandafter\def\csname PY@tok@gp\endcsname{\let\PY@bf=\textbf\def\PY@tc##1{\textcolor[rgb]{0.00,0.00,0.50}{##1}}}
\expandafter\def\csname PY@tok@go\endcsname{\def\PY@tc##1{\textcolor[rgb]{0.53,0.53,0.53}{##1}}}
\expandafter\def\csname PY@tok@gt\endcsname{\def\PY@tc##1{\textcolor[rgb]{0.00,0.27,0.87}{##1}}}
\expandafter\def\csname PY@tok@err\endcsname{\def\PY@bc##1{\setlength{\fboxsep}{0pt}\fcolorbox[rgb]{1.00,0.00,0.00}{1,1,1}{\strut ##1}}}
\expandafter\def\csname PY@tok@kc\endcsname{\let\PY@bf=\textbf\def\PY@tc##1{\textcolor[rgb]{0.00,0.50,0.00}{##1}}}
\expandafter\def\csname PY@tok@kd\endcsname{\let\PY@bf=\textbf\def\PY@tc##1{\textcolor[rgb]{0.00,0.50,0.00}{##1}}}
\expandafter\def\csname PY@tok@kn\endcsname{\let\PY@bf=\textbf\def\PY@tc##1{\textcolor[rgb]{0.00,0.50,0.00}{##1}}}
\expandafter\def\csname PY@tok@kr\endcsname{\let\PY@bf=\textbf\def\PY@tc##1{\textcolor[rgb]{0.00,0.50,0.00}{##1}}}
\expandafter\def\csname PY@tok@bp\endcsname{\def\PY@tc##1{\textcolor[rgb]{0.00,0.50,0.00}{##1}}}
\expandafter\def\csname PY@tok@fm\endcsname{\def\PY@tc##1{\textcolor[rgb]{0.00,0.00,1.00}{##1}}}
\expandafter\def\csname PY@tok@vc\endcsname{\def\PY@tc##1{\textcolor[rgb]{0.10,0.09,0.49}{##1}}}
\expandafter\def\csname PY@tok@vg\endcsname{\def\PY@tc##1{\textcolor[rgb]{0.10,0.09,0.49}{##1}}}
\expandafter\def\csname PY@tok@vi\endcsname{\def\PY@tc##1{\textcolor[rgb]{0.10,0.09,0.49}{##1}}}
\expandafter\def\csname PY@tok@vm\endcsname{\def\PY@tc##1{\textcolor[rgb]{0.10,0.09,0.49}{##1}}}
\expandafter\def\csname PY@tok@sa\endcsname{\def\PY@tc##1{\textcolor[rgb]{0.73,0.13,0.13}{##1}}}
\expandafter\def\csname PY@tok@sb\endcsname{\def\PY@tc##1{\textcolor[rgb]{0.73,0.13,0.13}{##1}}}
\expandafter\def\csname PY@tok@sc\endcsname{\def\PY@tc##1{\textcolor[rgb]{0.73,0.13,0.13}{##1}}}
\expandafter\def\csname PY@tok@dl\endcsname{\def\PY@tc##1{\textcolor[rgb]{0.73,0.13,0.13}{##1}}}
\expandafter\def\csname PY@tok@s2\endcsname{\def\PY@tc##1{\textcolor[rgb]{0.73,0.13,0.13}{##1}}}
\expandafter\def\csname PY@tok@sh\endcsname{\def\PY@tc##1{\textcolor[rgb]{0.73,0.13,0.13}{##1}}}
\expandafter\def\csname PY@tok@s1\endcsname{\def\PY@tc##1{\textcolor[rgb]{0.73,0.13,0.13}{##1}}}
\expandafter\def\csname PY@tok@mb\endcsname{\def\PY@tc##1{\textcolor[rgb]{0.40,0.40,0.40}{##1}}}
\expandafter\def\csname PY@tok@mf\endcsname{\def\PY@tc##1{\textcolor[rgb]{0.40,0.40,0.40}{##1}}}
\expandafter\def\csname PY@tok@mh\endcsname{\def\PY@tc##1{\textcolor[rgb]{0.40,0.40,0.40}{##1}}}
\expandafter\def\csname PY@tok@mi\endcsname{\def\PY@tc##1{\textcolor[rgb]{0.40,0.40,0.40}{##1}}}
\expandafter\def\csname PY@tok@il\endcsname{\def\PY@tc##1{\textcolor[rgb]{0.40,0.40,0.40}{##1}}}
\expandafter\def\csname PY@tok@mo\endcsname{\def\PY@tc##1{\textcolor[rgb]{0.40,0.40,0.40}{##1}}}
\expandafter\def\csname PY@tok@ch\endcsname{\let\PY@it=\textit\def\PY@tc##1{\textcolor[rgb]{0.25,0.50,0.50}{##1}}}
\expandafter\def\csname PY@tok@cm\endcsname{\let\PY@it=\textit\def\PY@tc##1{\textcolor[rgb]{0.25,0.50,0.50}{##1}}}
\expandafter\def\csname PY@tok@cpf\endcsname{\let\PY@it=\textit\def\PY@tc##1{\textcolor[rgb]{0.25,0.50,0.50}{##1}}}
\expandafter\def\csname PY@tok@c1\endcsname{\let\PY@it=\textit\def\PY@tc##1{\textcolor[rgb]{0.25,0.50,0.50}{##1}}}
\expandafter\def\csname PY@tok@cs\endcsname{\let\PY@it=\textit\def\PY@tc##1{\textcolor[rgb]{0.25,0.50,0.50}{##1}}}

\def\PYZbs{\char`\\}
\def\PYZus{\char`\_}
\def\PYZob{\char`\{}
\def\PYZcb{\char`\}}
\def\PYZca{\char`\^}
\def\PYZam{\char`\&}
\def\PYZlt{\char`\<}
\def\PYZgt{\char`\>}
\def\PYZsh{\char`\#}
\def\PYZpc{\char`\%}
\def\PYZdl{\char`\$}
\def\PYZhy{\char`\-}
\def\PYZsq{\char`\'}
\def\PYZdq{\char`\"}
\def\PYZti{\char`\~}
% for compatibility with earlier versions
\def\PYZat{@}
\def\PYZlb{[}
\def\PYZrb{]}
\makeatother


    % Exact colors from NB
    \definecolor{incolor}{rgb}{0.0, 0.0, 0.5}
    \definecolor{outcolor}{rgb}{0.545, 0.0, 0.0}



    
    % Prevent overflowing lines due to hard-to-break entities
    \sloppy 
    % Setup hyperref package
    \hypersetup{
      breaklinks=true,  % so long urls are correctly broken across lines
      colorlinks=true,
      urlcolor=urlcolor,
      linkcolor=linkcolor,
      citecolor=citecolor,
      }
    % Slightly bigger margins than the latex defaults
    
    \geometry{verbose,tmargin=1in,bmargin=1in,lmargin=1in,rmargin=1in}
    
    

    \begin{document}
    
    
    \maketitle
    
    

    
    \section{CS 61A: The Structure and Interpretation of Computer
Programs}\label{cs-61a-the-structure-and-interpretation-of-computer-programs}

    \subsection{Miscellaneous Resources}\label{miscellaneous-resources}

    \subsubsection{Tree Recursion}\label{tree-recursion}

Check out this link for more tree recursion practice:
https://practice.geeksforgeeks.org/topic-tags

\begin{enumerate}
\def\labelenumi{\arabic{enumi}.}
\item
  Given a number n, generate all distinct ways to write n as the sum of
  positive integers. For example, with n = 4, the options are 4, 3 + 1,
  2 + 2, 2 + 1 + 1, and 1 + 1 + 1 + 1.
\item
  Write a recursive function that checks whether a string is a
  palindrome (a palindrome is a string that's the same when reads
  forwards and backwards.)
\item
  Write a procedure that computes elements of Pascal's triangle by means
  of a recursive process. Define the procedure pascal(row, column) which
  takes a row and a column, and finds the value at that position in the
  triangle.
\item
  Consider the subset\_sum problem: you are given a list of integers and
  a number k. Is there a subset of the list that adds up to k?
\item
  We will now write one of the faster sorting algorithms commonly used,
  named Mergesort. Merge sort works like this:

  \begin{enumerate}
  \def\labelenumii{\alph{enumii}.}
  \tightlist
  \item
    If there's only one (or zero) item(s) in the sequence, it's already
    sorted!
  \item
    If there's more than one item, then we can split the sequence in
    half, sort each half recursively, then merge the results (using the
    merge procedure from earlier in the notes). The result will be a
    sorted sequence.
  \end{enumerate}
\end{enumerate}

Using the described algorithm, write a function mergesort(s) that takes
an unsorted sequence s and sorts it.

\begin{enumerate}
\def\labelenumi{\arabic{enumi}.}
\setcounter{enumi}{5}
\tightlist
\item
  Complete the definition of towers\_of\_hanoi which prints out the
  steps to solve this puzzle for any number of n disks starting from the
  start rod and moving them to the end rod.
\end{enumerate}

\begin{Shaded}
\begin{Highlighting}[]
\KeywordTok{def}\NormalTok{ towers_of_hanoi(n, start, end):}
\NormalTok{    …}

\KeywordTok{def}\NormalTok{ move_disk(start, end):}
\NormalTok{    …}
\end{Highlighting}
\end{Shaded}

\begin{enumerate}
\def\labelenumi{\arabic{enumi}.}
\setcounter{enumi}{6}
\tightlist
\item
  Implement Binary Search recursively.
\end{enumerate}

    \subsubsection{The Link Class (given on
exam)}\label{the-link-class-given-on-exam}

\begin{Shaded}
\begin{Highlighting}[]
\KeywordTok{class}\NormalTok{ Link: }
    \KeywordTok{def} \FunctionTok{__init__}\NormalTok{(}\VariableTok{self}\NormalTok{, first, rest}\OperatorTok{=}\NormalTok{empty):}
        \ControlFlowTok{assert}\NormalTok{ rest }\KeywordTok{is}\NormalTok{ Link.empty }\KeywordTok{or} \BuiltInTok{isinstance}\NormalTok{(rest, Link) }
        \VariableTok{self}\NormalTok{.first }\OperatorTok{=}\NormalTok{ first}
        \VariableTok{self}\NormalTok{.rest }\OperatorTok{=}\NormalTok{ rest}
    
\KeywordTok{def} \FunctionTok{__repr__}\NormalTok{(}\VariableTok{self}\NormalTok{):}
    \ControlFlowTok{if} \VariableTok{self}\NormalTok{.rest }\KeywordTok{is}\NormalTok{ Link.empty:             }
        \ControlFlowTok{return} \StringTok{"Link(}\SpecialCharTok{\{\}}\StringTok{)"}\NormalTok{.}\BuiltInTok{format}\NormalTok{(}\VariableTok{self}\NormalTok{.first)                }
    \ControlFlowTok{else}\NormalTok{:                   }
        \ControlFlowTok{return} \StringTok{"Link(}\SpecialCharTok{\{\}}\StringTok{, }\SpecialCharTok{\{\}}\StringTok{)"}\NormalTok{.}\BuiltInTok{format}\NormalTok{(}\VariableTok{self}\NormalTok{.first, }\VariableTok{self}\NormalTok{.rest)}
\end{Highlighting}
\end{Shaded}

\subsubsection{The Link Class
(customized)}\label{the-link-class-customized}

\begin{Shaded}
\begin{Highlighting}[]
\KeywordTok{class}\NormalTok{ Link:}
    \KeywordTok{def} \FunctionTok{__init__}\NormalTok{(}\VariableTok{self}\NormalTok{, first }\OperatorTok{=} \VariableTok{None}\NormalTok{, rest }\OperatorTok{=} \VariableTok{None}\NormalTok{):}
        \VariableTok{self}\NormalTok{.first }\OperatorTok{=}\NormalTok{ first}
        \VariableTok{self}\NormalTok{.rest }\OperatorTok{=}\NormalTok{ rest}
        
    \KeywordTok{def} \FunctionTok{__repr__}\NormalTok{(}\VariableTok{self}\NormalTok{):}
        \CommentTok{# empty link ---> None }
        \ControlFlowTok{if} \VariableTok{self}\NormalTok{.first }\OperatorTok{==} \VariableTok{None}\NormalTok{:}
            \ControlFlowTok{return} \StringTok{"()"}
        \ControlFlowTok{else}\NormalTok{:}
            \ControlFlowTok{return} \StringTok{'Link'} \OperatorTok{+} \StringTok{'('} \OperatorTok{+} \BuiltInTok{str}\NormalTok{(}\VariableTok{self}\NormalTok{.first) }\OperatorTok{+} \StringTok{","} \OperatorTok{+} \VariableTok{self}\NormalTok{.rest.}\FunctionTok{__repr__}\NormalTok{() }\OperatorTok{+} \StringTok{')'}
            
    \KeywordTok{def}\NormalTok{ multiply(}\VariableTok{self}\NormalTok{, n):}
        \ControlFlowTok{if} \VariableTok{self}\NormalTok{.rest.first }\OperatorTok{==} \VariableTok{None}\NormalTok{:}
            \VariableTok{self}\NormalTok{.first }\OperatorTok{=}\NormalTok{ n }\OperatorTok{*} \VariableTok{self}\NormalTok{.first}
        \ControlFlowTok{else}\NormalTok{:}
            \VariableTok{self}\NormalTok{.first }\OperatorTok{=}\NormalTok{ n }\OperatorTok{*} \VariableTok{self}\NormalTok{.first}
            \VariableTok{self}\NormalTok{.rest.multiply(n)}
            
    \KeywordTok{def}\NormalTok{ a_multiply(}\VariableTok{self}\NormalTok{, n):}
        \CommentTok{# if self == Link.empty}
        \ControlFlowTok{if} \VariableTok{self}\NormalTok{.rest.first }\OperatorTok{==} \VariableTok{None}\NormalTok{:}
            \ControlFlowTok{return}\NormalTok{ Link(}\VariableTok{self}\NormalTok{.first }\OperatorTok{*}\NormalTok{ n)}
        \ControlFlowTok{else}\NormalTok{:}
            \ControlFlowTok{return}\NormalTok{ Link(}\VariableTok{self}\NormalTok{.first }\OperatorTok{*}\NormalTok{ n, }\VariableTok{self}\NormalTok{.rest.a_multiply(n))}
            
    \KeywordTok{def}\NormalTok{ skip_node(}\VariableTok{self}\NormalTok{, n }\OperatorTok{=} \DecValTok{1}\NormalTok{):}
        \ControlFlowTok{if} \VariableTok{self}\NormalTok{.first }\OperatorTok{==} \VariableTok{None}\NormalTok{:}
            \ControlFlowTok{return}\NormalTok{ Link()}
        \ControlFlowTok{elif} \VariableTok{self}\NormalTok{.rest.first }\OperatorTok{==} \VariableTok{None}\NormalTok{:}
            \ControlFlowTok{return}\NormalTok{ Link(}\VariableTok{self}\NormalTok{.first, Link())}
        \ControlFlowTok{else}\NormalTok{:}
            \ControlFlowTok{return}\NormalTok{ Link(}\VariableTok{self}\NormalTok{.first, }\VariableTok{self}\NormalTok{.rest.rest.skip_node(n))}
        
        
\NormalTok{x }\OperatorTok{=}\NormalTok{ Link(}\DecValTok{5}\NormalTok{, Link(}\DecValTok{7}\NormalTok{, Link(}\DecValTok{9}\NormalTok{, Link(}\DecValTok{6}\NormalTok{, Link(}\DecValTok{8}\NormalTok{, Link())))))}
\NormalTok{y }\OperatorTok{=}\NormalTok{ [}\DecValTok{5}\NormalTok{,}\DecValTok{7}\NormalTok{,}\DecValTok{9}\NormalTok{,}\DecValTok{6}\NormalTok{]}
\BuiltInTok{print}\NormalTok{(x.a_multiply(}\DecValTok{3}\NormalTok{))}
\BuiltInTok{print}\NormalTok{(x)}
\BuiltInTok{print}\NormalTok{(x.skip_node())}
\end{Highlighting}
\end{Shaded}

\subsubsection{Additional Link Problems}\label{additional-link-problems}

Note 1: The term `list' refers to linked list. Note 2: You may use
methods you have created in the previous problems to support your code
for later problems.

Create a method that\ldots{}\\
1. adds an element to the beginning of the list (mutator) variant:
deletes an element from the beginning and returns it 2. adds an element
to the end of the list (mutator) variant: deletes an element from the
end and returns it 3. returns a new linked list that skips `n' nodes
from the original linked list 4. counts the number of times an element
occurs in the list 5. `deletes' all instances of an element from the
list 6. insert element at the nth position (mutator) 7. takes a sorted
list and inserts an element in the sorted position (mutator) 8. given a
list, returns a new list that contains all of the nodes in sorted order
9. appends the elements of another linked list onto the end of the
original one 10. find the length of the longest consecutive sequence of
elements that is strictly increasing 11. remove all duplicates from a
sorted list 12. reverse the order of the elements in the list 13. Refer
to this Stanford link and this interview problems link for more
problems.

    \subsubsection{OOP}\label{oop}

Source:
https://stackoverflow.com/questions/245800/oop-problems-to-use-for-coding-tests-during-interviews

\begin{enumerate}
\def\labelenumi{\arabic{enumi})}
\item
  Create model classes that will properly represent the following
  constructs: Define a Shape object, where the object is any two
  dimensional figure, and has the following characteristics: a name, a
  perimeter, and a surface area. Define a Circle, retaining and
  accurately outputting the values of the aforementioned characteristics
  of a Shape. Define a Triangle. This time, the name of the triangle
  should take into account if it is equilateral (all 3 sides are the
  same length), isosceles (only 2 sides are the same length), or scalene
  (no 2 sides are the same). Continue with the quadrilaterals!
\item
  We need to control access to a customer web site. each customer may
  have one or more people to access the site different people from
  different customers may be able to view different parts of the site
  the same person may work for more than one customer customers want to
  manage permissions based on the person, department, team, or project
  Design a solution for this using object-oriented technique.
\item
  (Are you smarter than a CS Grad?) Write a program that prints the
  numbers from 1 to 100. But for multiples of three print "Fizz" instead
  of the number and for the multiples of five print "Buzz". For numbers
  which are multiples of both three and five print "FizzBuzz."
\item
  Make a rough class diagram of the game Pacman. What classes would you
  need? What class / instance variables are necessary? Would some
  classes depend on the others?
\end{enumerate}

    \subsubsection{Tree Data Structures}\label{tree-data-structures}

Source:
https://www.careercup.com/page?pid=trees-and-graphs-interview-questions

Directions: For each problem, write a new method for the Tree class. You
can only use the methods already given or a method from the problems
above.

\begin{Shaded}
\begin{Highlighting}[]
\KeywordTok{class}\NormalTok{ Tree:}
    
    \KeywordTok{def} \FunctionTok{__init__}\NormalTok{(}\VariableTok{self}\NormalTok{, label, branches):}
        \VariableTok{self}\NormalTok{.label }\OperatorTok{=}\NormalTok{ label}
        \VariableTok{self}\NormalTok{.branches }\OperatorTok{=}\NormalTok{ branches}

    \KeywordTok{def}\NormalTok{ is_leaf(}\VariableTok{self}\NormalTok{):}
        \ControlFlowTok{if} \KeywordTok{not} \VariableTok{self}\NormalTok{.branches:}
            \ControlFlowTok{return} \VariableTok{True}
        \ControlFlowTok{return} \VariableTok{False}

    \KeywordTok{def}\NormalTok{ branches(}\VariableTok{self}\NormalTok{):}
        \ControlFlowTok{return} \VariableTok{self}\NormalTok{.branches}

    \KeywordTok{def}\NormalTok{ label(}\VariableTok{self}\NormalTok{):}
        \ControlFlowTok{return} \VariableTok{self}\NormalTok{.label}
\end{Highlighting}
\end{Shaded}

\begin{enumerate}
\def\labelenumi{\arabic{enumi}.}
\tightlist
\item
  Minimum tree height (i.e. shortest branch from root node)
\item
  Largest integer value in the tree
\item
  Are two trees equal?
\item
  Traverse all of the elements of the tree in whatever order you
  want\ldots{} using iteration
\item
  Find the minimum sum of any branch from root to leaf.
\item
  Find the maximum sum among all elements in the tree (don't have to be
  in the same branch).
\item
  Given a Tree where each node contains an attribute e.g. color
  (R,G,B... etc), find the subtree with the maximum number of
  attributes.
\end{enumerate}

\begin{Shaded}
\begin{Highlighting}[]
\NormalTok{Input: }
\NormalTok{G }
\OperatorTok{/} \OperatorTok{\textbackslash{}} 
\NormalTok{B  R }
\OperatorTok{/} \OperatorTok{\textbackslash{}} \OperatorTok{/} \OperatorTok{\textbackslash{}} 
\NormalTok{B B R R }
\OperatorTok{/} \OperatorTok{\textbackslash{}} \OperatorTok{/} \OperatorTok{\textbackslash{}} 
\NormalTok{B R R R }

\NormalTok{Output: }
\NormalTok{R }
\OperatorTok{/} \OperatorTok{\textbackslash{}} 
\NormalTok{R R }
\OperatorTok{\textbackslash{}} \OperatorTok{/} \OperatorTok{\textbackslash{}} 
\NormalTok{R R R}
\end{Highlighting}
\end{Shaded}


    % Add a bibliography block to the postdoc
    
    
    
    \end{document}
